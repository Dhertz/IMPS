\documentclass[11pt]{report}

\usepackage[latin1]{inputenc}
\usepackage{tikz}
\usetikzlibrary{shapes,arrows}
\usepackage{chngpage}

\begin{document}
\setlength{\topmargin}{-1cm} %Was a bit bottom heavy in my opinon

\title{The IMPS Debugger}
\author{An extension to the IMPS system\\written by Owen Davies, Daniel Hertz and Charlie Hothersall-Thomas}
\date{ \today}

\maketitle
\section*{The IMPS Debugger} %asterix stops enumerating sections

The IMPS debugger allows the user to debug an IMPS source file. It takes one 
command line argument - the source file that the user wishes to debug. For 
example, to debug the source file \textbf{example.s} one would need to write 
the following command:
\\[1ex]
\texttt{\% ./debug example.s}
\\[1ex]
The debugger is a standalone tool, which shares methods from the IMPS emulator 
and IMPS assembler. We created a seperate file, \textbf{utils.c}, which contains 
shared methods for the emulator, assembler and our debugger.

\subsection*{Example Usage}

Once the program is started, the input file is read and a symbol table is 
constructed. The source file is then assembled into IMPS binary. The user is 
then presented with a prompt:
\\[1ex]
\texttt{\% ./debug example.s\\
		Reading input file example.s...done.\\
		Building symbol table...done.\\
		Welcome to the IMPS debugger. Type "h" or "help" for information on commands.}
\\[1ex]
From here the user can add breakpoints to halt execution at a certain line, set 
the value of any of the 31 registers or any memory address, and start executing 
the program with the \texttt{run} command.\\
If no breakpoints are set, then the program will run through normally and exit, 
as if it had been assembled and emulated normally. If one or more breakpoints 
have been set, then the program will execute line by lnie until the first 
breakpoint is hit. Say the user set a breakpoint at line 8:
\\[1ex]
\texttt{Breakpoint at line 8 reached. What do you want to do?}
\\[1ex]
Now the program has reached a breakpoint, the user can move to the next line, 
continue to the next breakpoint (or to the end of the program if there are no 
more breakpoints), or print or set the values of registers and memory 
addressses.
\newpage
The following flow diagram shows the basic run-down of the debugger:

\begin{center}
	% Define block styles
	\tikzstyle{decision} = [diamond, draw, fill=white!20, 
		text width=4.5em, text badly centered, node distance=3cm, inner sep=0pt]
	\tikzstyle{block} = [rectangle, draw, fill=white!20, 
		text width=5em, text centered, rounded corners, minimum height=4em]
	\tikzstyle{line} = [draw, -latex']
		
	\begin{tikzpicture}[scale=0.40, node distance = 2cm, auto]
		\node [block] (start) {Build symbol table and assemble};
		\node [block, below of=start] (break) {Set breakpoints};
		\node [block, below of=break] (execute) {Emulate line};
		\node [decision, below of=execute] (halt) {Have we reached halt?};
		\node [decision, below of=halt, node distance=4cm] (breakpoint) {Have we reached a breakpoint?};
		\node [block, left of=halt, node distance=3cm] (next) {Next line};
		\node [block, right of=halt, node distance=3.5cm, fill=red!20] (stop) {Stop Execution};
		\node [decision, below of=breakpoint, node distance=4cm] (CLI) {Command line Interface};
		\node [block, left of=breakpoint, node distance=6cm] (step) {Set Break for next line};
		%\node [block, right of=CLI, node distance=6cm] (print) {Print registers or Address};
		\path [line] (start) -- (break);
		\path [line] (break) -- (execute);
		\path [line] (execute) -- (halt);
		\path [line] (halt) -- node {no}(breakpoint);
		\path [line] (next) |- (execute);
		\path [line] (halt) -- node {yes}(stop);
		\path [line] (breakpoint) -| node {no}(next);
		\path [line] (CLI) -| node [above of=CLI, left] {Continue}(next);
		\path [line] (CLI) -| node {Next}(step);
		\path [line] (step) |- (next);
		\path [line] (breakpoint) -- node {yes}(CLI);
		\path [line] (CLI) -| node [above of=CLI, left] {Quit}(stop);
		%\path [line] (CLI) -- node [near end] {Print}(print);
		%\path [line] (print) |- (CLI.north);
	\end{tikzpicture}
\end{center}

\subsection*{Table of Commands}

The table below lists all the commands available in the debugger, as well as the 
command number that assosiates the command with its short version (using a data 
structure to map the command and its short version to the same command number), 
and a short description of what the command does.
\\
\begin{adjustwidth}{-.5in}{-.5in} 
	\begin{center} 
	\begin{tabular}{ | l | c | c | p{5cm} |}
	\hline
	Command & Number & Short & Description \\ \hline
	quit & 0 & q & Halt the execution and exit the program \\ \hline
	break $<$line$>$ & 1 & b $<$line$>$ & Set a breakpoint at line $<$line$>$ \\ \hline
	run & 2 & r & Start running the program \\ \hline
	next & 3 & n & Execute the next line of code \\ \hline
	continue & 4 & c & Continue execution to the next break point \\ \hline
	print & 5 & p & Print the contents of all registers \\ \hline
	help & 6 & h & Print the descriptions in this table \\ \hline
	setReg $<$regno$>$ $<$value$>$ & 7 & sR & Set the value of \$$<$regno$>$ to $<$value$>$ \\ \hline
	setAddr $<$addr$>$ $<$value$>$ & 8 & sA & Set the value of mem[$<$addr$>$] to $<$value$>$ \\ \hline
	printReg $<$regno$>$ & 7 & pR & Print the value of \$$<$regno$>$ \\ \hline
	printAddr $<$addr$>$ & 8 & pA & Print the value of mem[$<$addr$>$] \\ \hline
	\end{tabular}
	\end{center}
\end{adjustwidth}

\subsection*{Testing}

The debugger was tested while we were making it by successfully printing out states that we knew to be correct. 
This helped us quickly locate and correct unexpected behaviour when we experienced it, instead of trawling through 
lines of code. On occasion the gnome debugger was used, and debug-ception occurred. This enabled us to quickly 
locate where our code was failing, as we could look at the stack and line number calls. GDB was thusly used
for inspiration in the design of our CLI and what commands should be included in the IMPS debugger. When 
the program was completed all 4 of the assembly files were used to test every function of the debugger, confirming
each feature was operational and there was no unexpected behaviour.

\section*{Programming in a Group}

\begin{center}\textit{Producing a software suite for an imaginary language has never been so fulfilling.} \end{center}
Programming in the group has been a good experience for all of us, as we have 
learnt equal parts group coding skills and C. We mainly divided up by doing 
what we were interested in, but always checking and making edits to everyone 
else's work, so the whole group knew the progress that had been made. One of 
the major learning points for one of the group members was the use of git, which 
he was very sceptical about in the beginning, as he didn't really understand the 
use case. The other two members have now taught the importance of version 
control, but admit that we were not doing it quite right. We had a shared 
repository, on which we all had our own branches and pushed and pulled from, 
and the master was holy, and only to be touched if we are sure. This is an 
antiquated model of Subversion. We learnt where git really excels -  its 
distributed model. If we each had our own repository, each with our master, 
then we could branch of and do some experimental work on our own and merge it 
into out master branch. If then the rest of the group members wanted by change, 
and we agreed it was good, then they would pull from the individuals repo. This 
makes git much more powerful and easier to use, with nicer merges. Our methods 
communication went through a number of iterations, first it was mostly 
in-person, but either someone would be left out of the loop, or if we were 
programming, have not much to do. We then decided to split up tasks a bit more, 
and two of us pair-programming on a larger project, while in parallel the other 
was working on something related. This worked for some time, but as we came 
under more time pressure, the natural thing came to us. Facebook group chat, 
each working in parallel on something different, and then everyone checking each 
others work. This way, everyone knew what everyone else was doing and we could 
quickly ask any questions about the design or methodology the others were happy 
to use. Another powerful tool we used was gitHub to track our commits. This 
means whenever we happen to meet, the code is available to us online, and we 
could discuss changes or possible design considerations, as well as seeing easy 
'diffs' of commits.\\
All in all we have really enjoyed working together and it has made us better people for it. Now I can't wait for summer!
\end{document}
