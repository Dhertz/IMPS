\documentclass[11pt]{report}

\usepackage[latin1]{inputenc}
\usepackage{tikz}
\usetikzlibrary{shapes,arrows}
\usepackage{chngpage}

\begin{document}
\setlength{\topmargin}{-1cm}
\setlength{\parindent}{0in}

\title{The IMPS Debugger}
\author{An extension to the IMPS system\\written by Owen Davies, Daniel Hertz and Charlie Hothersall-Thomas}
\date{ \today}

\maketitle
\section*{The IMPS Debugger}

The IMPS debugger allows the user to debug an IMPS source file. It takes one 
command line argument - the source file that the user wishes to debug. For 
example, to debug the source file \textbf{example.s} one would need to write 
the following command:
\\[2ex]
\texttt{\% ./debug example.s}
\\[2ex]
The debugger is a standalone tool, which shares methods from the IMPS emulator 
and IMPS assembler. We created a seperate file, \textbf{utils.c}, which contains 
shared methods for the emulator, assembler and our debugger.

\subsection*{Example Usage}

Once the program is started, the input file is read and a symbol table is 
constructed. The source file is then assembled into IMPS binary. The user is 
then presented with a prompt:
\\[2ex]
\texttt{\% ./debug example.s\\
		Reading input file example.s... done.\\
		Building symbol table... done.\\
		Welcome to the IMPS debugger. Type "h" or "help" for information on commands.}
\\[2ex]
From here the user can add breakpoints to halt execution at a certain line, set 
the value of any of the 32 registers or any memory address, and start executing 
the program with the \texttt{run} command. If no breakpoints are set, then the
program will run through normally and exit, as if it had been assembled and
emulated normally. If one or more breakpoints have been set, then the program
will execute line by line until the first breakpoint is hit. Say the user set a
breakpoint at line 8:
\\[2ex]
\texttt{Breakpoint at line 8 reached. What do you want to do?}
\\[2ex]
Now the program has reached a breakpoint, the user can move to the next line, 
continue to the next breakpoint (or to the end of the program if there are no 
more breakpoints), or print or set the values of registers and memory 
addressses.

\newpage

The following flow diagram shows the basic run-down of the debugger:

\begin{center}
	% Define block styles
	\tikzstyle{decision} = [diamond, draw, fill=white!20, 
		text width=4.5em, text badly centered, node distance=3cm, inner sep=0pt]
	\tikzstyle{block} = [rectangle, draw, fill=white!20, 
		text width=5em, text centered, rounded corners, minimum height=4em]
	\tikzstyle{line} = [draw, -latex']
		
	\begin{tikzpicture}[scale=0.40, node distance = 2cm, auto]
		\node [block] (start) {Build symbol table and assemble};
		\node [block, below of=start] (break) {Set breakpoints};
		\node [block, below of=break] (execute) {Emulate line};
		\node [decision, below of=execute] (halt) {Have we reached halt?};
		\node [decision, below of=halt, node distance=4cm] (breakpoint) {Have we reached a breakpoint?};
		\node [block, left of=halt, node distance=3cm] (next) {Next line};
		\node [block, right of=halt, node distance=3.5cm, fill=red!20] (stop) {Stop Execution};
		\node [decision, below of=breakpoint, node distance=4cm] (CLI) {Command Line Interface};
		\node [block, left of=breakpoint, node distance=6cm] (step) {Set Break for next line};
		%\node [block, right of=CLI, node distance=6cm] (print) {Print registers or Address};
		\path [line] (start) -- (break);
		\path [line] (break) -- (execute);
		\path [line] (execute) -- (halt);
		\path [line] (halt) -- node {no}(breakpoint);
		\path [line] (next) |- (execute);
		\path [line] (halt) -- node {yes}(stop);
		\path [line] (breakpoint) -| node {no}(next);
		\path [line] (CLI) -| node [above of=CLI, left] {Continue}(next);
		\path [line] (CLI) -| node {Next}(step);
		\path [line] (step) |- (next);
		\path [line] (breakpoint) -- node {yes}(CLI);
		\path [line] (CLI) -| node [above of=CLI, left] {Quit}(stop);
		%\path [line] (CLI) -- node [near end] {Print}(print);
		%\path [line] (print) |- (CLI.north);
	\end{tikzpicture}
\end{center}

\subsection*{Table of Commands}

The table below lists all the of commands available in the debugger and a short description of what each command does.\\

\begin{adjustwidth}{-.5in}{-.5in} 
	\begin{center} 
	\begin{tabular}{ | l | c | p{5cm} |}
	\hline
	Command & Short & Description \\ \hline
	quit & q & Halt the execution and exit the program \\ \hline
	break $<$line$>$ & b $<$line$>$ & Set a breakpoint at line $<$line$>$ \\ \hline
	run & r & Start running the program \\ \hline
	next & n & Execute the next line of code \\ \hline
	continue & c & Continue execution to the next break point \\ \hline
	print & p & Print the program counter and contents of all registers \\ \hline
	help & h & Print the list of supported debugger commands \\ \hline
	setReg $<$regno$>$ $<$value$>$ & sR & Set the value of \$$<$regno$>$ to $<$value$>$ \\ \hline
	setAddr $<$addr$>$ $<$value$>$ & sA & Set the value of mem[$<$addr$>$] to $<$value$>$ \\ \hline
	printReg $<$regno$>$ & pR & Print the value of \$$<$regno$>$ \\ \hline
	printAddr $<$addr$>$ & pA & Print the value of mem[$<$addr$>$] \\ \hline
	\end{tabular}
	\end{center}
\end{adjustwidth}

\subsection*{Testing}

Testing of the debugger was ongoing throughout development, using carefully-written
source files to cover every concievable edge case, whilst ensuring that the debugger
always behaved as expected and that any outputted values were correct. The GNU
Project Debugger, \textbf{gdb}, also proved useful during development. This allowed
us to use the stack trace to quickly locate and correct problems as they were
experienced, rather than having to search through our source code. We also used
gdb's interface as inspiration when designing the CLI for our debugger.

\newpage

The primary problem encountered when writing the debugger was caused by empty
lines in source files. These are discarded by the assembler, yet when the user
inputs a line number to set a breakpoint, the empty lines in the original source
file still need to be considered when determining where to set the breakpoint.
Our solution to this was to scan through the source file looking for blank lines,
generating a map (stored as an array) of assembly line numbers to 'human' line
numbers.

\section*{Programming in a Group}

Group programming has been a fulfilling experience for us, as we have each
improved our group work skills whilst also becoming more proficient in C. All
three members of the group had no prior experience of collaborative programming,
but it worked well and we each feel that our teamwork and communication skills
are better as a result.
\\[2ex]
The work was divided roughly equally; we did our best to delegate each task to the team
member who was most interested in completing it, but this was not always possible. It 
was important that every one knew about progress made by the remainder of the group, so we
consistently checked each other's work throughout the process, suggesting
alterations and improvements where possible. We also tried to ensure that all large design
issues were addressed as a group, to make sure that the most appropriate decision was
made, including the file structure and algorithms used.
\\[2ex]
The group project was also a good exercise in version control, using git to make
it easier to track and combine every one's work. We later realised that we
weren't quite using git correctly, although we have learned for next time. We
had a shared repository and each had our own branch, using the master branch
only after discussing the commit with the entire group. This is an antiquated
model of Subversion. git really excels with its distributed model - if we each
had our own repository (with its own master branch) then we could branch out for
some solo experimental work, merge into our individual master branch, and then
if the rest of the group agreed with the work, it could be pulled into the others' repositories 
and merged. This makes git much more powerful and easier to use, with nicer merges.

\newpage

To start with, we communicated primarily in person and worked individually, but
we soon realised that this was not the most efficient way, as people were either
being left out of the loop or with not enough work to do. We therefore decided to
split up tasks into smaller chunks, and introduced pair-programming on the larger
tasks while the other group member worked on something else. We found this new
system to be much more efficient, with the bonus that everybody knew what the
others were working on at any given time.
\end{document}
